\subsection{blaat}
    \includegraphics[]{'Uppaal Automata.pdf'}

Dit zijn al onze UPPAAL componenten.
\begin{itemize}
    \item ProgramInit zorgt ervoor dat alle componenten een nette beginstand hebben.
    \item Button staat voor een knop (noodknop of startknop). Een button kan pressed of released zijn.
    \item Lamp staat voor een LED of de UVlamp. Deze kan aan of uit staan.
    \item Conveyor is de lopende band die maar 1 kant op kan lopen. Deze kan dus 'aan' (running) of 'uit' (stopped) zijn.
    \item ConveyorBF kan heen en terug lopen. Er zijn dus staten voor back, forward en stopped nodig.
    \item Door is het model van een deur. Een deur kan open of dicht zijn, maar daarnaast ook nog bezig zijn met openen of sluiten. Als een geopende deur een open-signaal krijgt (of een gesloten deur een sluit-signaal), dan wordt dit verwerkt door meteen aan te geven dat de deur klaar is met openen (of sluiten).
    \item Sensor is een licht- of druksensor. Een sensor kan Sense of NoSense zijn - hij neemt iets waar of hij neemt niets waar.
    \item FallOff - dit is een speciaal geval. Het is niet echt een hardware-component (ondanks dat er wel LEDs aan en uit gaan ten gevolge van de staat van dit onderdeel). FallOff houdt bij hoeveel wafers van de band gevallen zijn. Zodra er vijf wafers verloren zijn gegaan 'crasht' ons systeem. (Dit is een noodtoestand)We houden dit op deze manier bij omdat de hoeveelheid verloren wafers van groot belang is voor de werking van het systeem.
\end{itemize}
Veel van de componenten hebben een onbekende beginstand, een gegeven dat gemodelleerd is door vanuit Init een stap te maken naar een willekeurige toestand.\\
Daarnaast wordt van veel componenten verwacht dat ze een signaal kunnen versturen (deze deur staat open!) - dit wordt gedaan door in elke  'stabiele' staat (de deur staat open) een loopje naar dezelfde staat te maken. Deze loop verstuurt een signaal over de gewenste channel.\\
Een element dat duidelijk afwezig is in ons model is de wafer. We hebben er expliciet voor gekozen om dit niet te modelleren, omdat we helemaal niets weten over de wafer, de enige informatie hierover krijgen we van de sensoren. Ook kunnen we de wafer zelf niet aansturen, dit gaat met behulp van andere onderdelen.
