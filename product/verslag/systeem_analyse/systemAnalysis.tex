\documentclass[]{report}
%% Gebruikte pakketten
\usepackage[dutch]{babel}
\usepackage{geometry}
\usepackage{graphicx}
\usepackage{amssymb}
\usepackage{epstopdf}
\usepackage{textcomp}
\usepackage[]{hyperref}
\usepackage{fancyhdr}

\title{Systeem analyse verslag UPPAAL}
\author{Groep 7 OGO 1.3 2006-2007}
\date{05-22-2007}


\begin{document}
  %%%%%%%%%%%%%%%%%%%%%%%%%%%%%%%%%%%%%%%%%%%%%%%%%%%%%%%%%%%%%%%%%
% Contents: The title page
% $Id: title.tex,v 1.2 2003/03/19 20:57:47 oetiker Exp $
%%%%%%%%%%%%%%%%%%%%%%%%%%%%%%%%%%%%%%%%%%%%%%%%%%%%%%%%%%%%%%%%%
\ifx\pdfoutput\undefined % We're not running pdftex
\else
\pdfbookmark{Titelblad}{title} \fi
\newlength{\centeroffset}
\setlength{\centeroffset}{-0.5\oddsidemargin}
\addtolength{\centeroffset}{0.5\evensidemargin}
%\addtolength{\textwidth}{-\centeroffset}
\thispagestyle{empty}
\vspace*{\stretch{1}}
\noindent\hspace*{\centeroffset}\makebox[0pt][l]{\begin{minipage}{\textwidth}
\flushright
{\Huge\bfseries Wafer stepper\\
                  <verslag> \\}
\noindent\rule[-1ex]{\textwidth}{5pt}\\[2.5ex]
\hfill\emph{\Large OGO 1.3: Groep 7}
\end{minipage}}

\vspace{\stretch{1}}
\noindent\hspace*{\centeroffset}\makebox[0pt][l]{\begin{minipage}{\textwidth}
\flushright
{\bfseries
door: \\ \newline
\begin{tabular}{ r | c}
  \multicolumn{2}{c}{ } \\
  Etienne van Delden & 0618959  \\
  Gijs Direks        & 0611093  \\
  Sanne Ernst        & 0588898  \\
  Bas Goorden        & 0598669  \\
  Stef Sijben        & 0607426  \\ 
  Coen van der Wel   & 0608467  \\

\end{tabular}\\ [3ex]}
<datum>
\end{minipage}}

%\addtolength{\textwidth}{\centeroffset}
\vspace{\stretch{2}}



\endinput

  \tableofcontents

    \chapter{Inleiding}
        Bij de systeem analyse controleren we of ons UPPAAL-model correct is. Om dit te controleren moeten
        we eerst defini\"eren wat we verstaan onder ,,correct''. Wij zullen dit defini\"eren met
        behulp van eigenschappen waaraan ons systeem dient te voldoen; wanneer dit het geval is, achten
        wij het systeem correct. Deze eigenschappen zullen we toelichten, vertalen naar UPPAAL-queries, en
        testen volgens het verificatiesysteem van UPPAAL zelf. De resultaten zullen we ook evalueren in dit
        document.


    \chapter{Belangrijke eigenschappen functionaliteit}
        Hieronder is beschreven aan welke eigenschappen ons systeem moet voldoen. Het vierde item kan echter niet
        gegarandeerd worden bij indrukken van de noodknop; met deze uitzondering zal bij de simulatie en verificatie
        rekening worden gehouden.
        \begin{enumerate}
            \item Er zijn nooit twee deuren tegelijk open.
            \item De lamp is alleen aan bij belichten van een wafer.
            \item Vlak voordat een wafer aan ,,zijn cyclus'' begint is deur 1 open en deur 2 dicht.
            \item Voor alle paden geldt dat een wafer binnen 20 seconden in een bak ligt of verloren is gegaan, mits er niet
                  op de noodknop wordt gedrukt.
            \item Het systeem crasht alleen wanneer het moet crashen.
        \end{enumerate}

    \chapter{Methodiek simuleren en verifi\"eren eigenschappen}
        Voor het simuleren van het systeem gebruiken wij de Simulator welke in UPPAAL beschikbaar is. Deze
        laten we random acties simuleren op de hoogste snelheid voor een duur van maximaal 5 minuten. Dit
        proces herhalen we enkele malen om er zeker van te zijn dat ons systeem inderdaad geen deadlocks bevat.\\
        \\
        Bij het verifi\"eren maken wij gebruik van z.g.n. queries, welke uitdrukken wat precies wij willen
        verifi\"eren. De nummering is gelijk aan die van de vorige lijst:
        \begin{enumerate}
            \item A[] (door1.Closed or door1.PseudoClosing or door2.Closed or door2.PseudoClosing)\\
                \emph{    Er moet altijd gelden dat er minstens 1 deur gesloten is. De toestandsnaam PseudoClosing
                            zou kunnen impliceren dat de deur aan het sluiten is; dit is echter een tussentoestand
                            in UPPAAL en in feite is de deur dan al gesloten.}
            \item A[] (lamp.On) imply (main.Burning)\\
                \emph{    Als de UV-lamp aanstaat, moet gelden dat er een wafer wordt gebrand.}
            \item A[] main.ReadyToLoadWafer imply (door1.Open and door2.Closed)\\
                \emph{    Voordat er een wafer wordt geladen, staat deur 1 open en is deur 2 gesloten. ``main'' is de naam
                van het hoofdprogramma.}
            \item A[] main.ReadyToLoadWafer imply (main.total $<$= 15)\\
                \emph{    Voordat er een nieuwe wafer wordt geladen en dus de vorige reeds klaar is, moet
                            gelden dat het aantal verstreken seconden niet hoger is dan 15 sinds laden van
                            die wafer.}
            \item A[] managecrash.Crashed or not deadlock\\
                \emph{    Het systeem is ofwel in een crash-toestand, of draait zoals normaal.}
        \end{enumerate}

    \chapter{Simulatie- en verificatieresultaten}
        Volgens de eerste beschreven methode hebben wij het systeem gesimuleerd. Formeel zijn de resultaten hieronder
        beschreven. Informeel komt het erop neer dat al onze simulaties goed zijn verlopen en we dus ons systeem als
        correct mogen beschouwen. Helaas komt het regelmatig voor (dankzij UPPAAL's testsysteem) dat de deuren vast-
        lopen en komt daardoor in een crash-toestand. Dit is echter de bedoeling wanneer een deur vastloopt, dus is
        een resultaat van deze aard een gewenst resultaat.\\
        \\
        \begin{tabular}{ c | l | c }
            Duur & Opmerkingen & Resultaat\\
            \hline
            9s & deur weigert -$>$ crash & succesvol\\
            2s & deur weigert -$>$ crash & succesvol\\
            1s & deur weigert -$>$ crash & succesvol\\
            5s & deur weigert -$>$ crash & succesvol\\
            4s & deur weigert -$>$ crash & succesvol\\
        \end{tabular}\\\\
        \\
        Volgens de tweede beschreven methode hebben wij het systeem geverifi\"eerd. Bij al onze opgegeven queries
        gaf het systeem het resultaat ,,Property is satisfied'', wat dus inhoudt dat in alle gevallen het systeem
        voldoet aan de opgegeven eigenschappen.\\

    \chapter{Conclusie}
        We kunnen dus concluderen dat ons systeem voldoet aan de gestelde eisen en eigenschappen.

\end{document}
