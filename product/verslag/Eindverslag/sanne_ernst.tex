\subsection{Persoonlijke evaluatie van Sanne Ernst}

Bij wiskunde heb ik modelleervakken gevolgd, daarentegen was OGO
voor mij nieuw. In plaats van werken in groepjes van twee moesten we
nu met zes man werken. Dit maakt het werken in een groep een stuk
gezelliger en je bent ook niet gebonden aan \'e\'en partner.
Dit was voor mij een nieuwe en ook prettige ervaring. \\

Wat voor mij totaal nieuw was, waren de wekelijkse vergaderingen.
Hierbij heb ik het belang van vergaderen geleerd, namelijk het
periodieke spreken van de opdrachtgever. Echter vond ik de
vergaderingen te formeel en over het algemeen te lang duren.\\

Voor OGO 1.3 had ik nog niet zoveel ervaring met het ontwerpen en
maken van programmatuur. Ik had wel al ervaring met het programmeren
van java-applicaties bij modelleren. Nu ben ik voor het eerst
begonnen met het maken van een model alvorens de applicatie zelf te
maken. Dit werkte zeer prettig, omdat ik nu een richtlijn had voor
het schrijven van code. Ook heb ik meer ervaring opgedaan met het
schrijven van assembler code.\\

Verder heb ik meegeholpen met het bouwen van het hardware-model. Dit
vond ik heel erg leuk om te doen, omdat dit een heel gepuzzel was.
De vorm van de stukjes maakt het soms lastig om te bepalen wat het
doel ervan is. Het werken met Fischertechnik deed mij denken aan
het spelen met Lego. Wat ik onder andere ontworpen heb zijn de lopende banden.\\

Hoewel we een moeilijke start hadden, werd het uiteindelijk een heel
geslaagde OGO.
