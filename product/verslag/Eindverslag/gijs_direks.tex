\subsection{Persoonlijke evaluatie van Gijs Direks}

Dit was mijn eerste OGO, mijns inziens ook een zeer geslaagde.
Na een paar opstartproblemen, vooral te wijten aan het feit dat
maar twee personen al eerder OGO gehad hebben, liep alles lekker
 door. We hebben met zes man hard en goed gewerkt aan dit project.\\
\\
Ik heb ook veel geleerd van deze OGO. Naast het omgaan met andere
mensen was er natuurlijk ook het technische aspect. Fischertechnik
is natuurlijk geen ontzettend hoogstaand materiaal en brengt menige
beperking met zich mee, maar dit levert ook uitdagingen die
overwonnen moeten worden. Enkele van deze problemen vergden 'nieuwe'
oplossingen, die een andere kijk met zich meebrachten.\\
\\
Ook de practicumprocessor heeft enkele kopzorgen gebaard: bij het
ontwerp van de elektrische interface had ik er bijvoorbeeld geen
rekening mee gehouden dat er pull-up weerstanden aan de poorten
zitten - dit had als effect dat ik de noodknop en de sensoren op een
ingangspoort en +5V aangesloten had. De processor las dan ook geen
bruikbare waarden uit.
Na een opmerking van Stef en wat solderen was dit toch weer snel verholpen.\\
\\
Al met al vond ik dit een zeer geslaagde OGO, het was een interessant probleem en ik heb met leuke mensen samengewerkt.
