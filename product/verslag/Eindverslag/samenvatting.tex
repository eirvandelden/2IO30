\begin{abstract}
\textbf{Tijdens dit OGO-project hebben we een waferstepper
nagemaakt. Een waferstepper is een machine die silicium wafers
belicht met UV-licht om er IC's van te maken. Dit belichten gebeurt
in een vacu\"{u}m, omdat UV-licht door de atmosfeer wordt
geabsorbeerd en omdat de lens kapot gaat wanneer deze aan de
buitenlucht wordt blootgesteld.}

\textbf{De waferstepper zelf hebben we gebouwd met behulp van
Fischertechnik. Hiermee zijn eenvoudig de echte loopbanden,
sluisdeuren, lamp en sensoren na te bootsen. We sturen de machine
aan met een processorbord met daarop een Siemens SAB-C504 processor.
Deze hebben we geprogrammeerd met assembleertaal. Voordat we de
processor geprogrammeerd hebben, hebben we een programmaontwerp
gemaakt en getest in UPPAAL.}

\end{abstract}
\clearpage
