\subsection{Persoonlijke evaluatie van Stef Sijben}

Deze OGO begon enigszins moeizaam, waarschijnlijk mede doordat in
onze groep slechts twee personen waren die al eerder een OGO hadden
gedaan. Na een aantal weken ging het echter beter en vanaf dat
moment zijn er weinig grote problemen meer geweest. De samenwerking
met de rest van de groep is in het algemeen goed verlopen.\\

Het gebruik van Fischertechniek bracht de nodige beperkingen met
zich mee, maar het was een leuke uitdaging om hier omheen te werken,
mede omdat we graag wilden proberen de hele waferstepper op \'e\'en
bordje te bouwen.

Daarnaast hebben we geleerd dat het bij een eventueel volgend
project handig is om de elektrische aansluitingen wat beter te
organiseren. Nu waren er namelijk nogal wat draden fout aan elkaar
gesoldeerd, waardoor het model niet goed werkte.\\

Ondanks dat ik al de nodige ervaring had met embedded systemen, heb
ik op dit gebied toch ook nog de nodige dingen geleerd omdat deze
processor op sommige gebieden nogal verschillend is van de
processoren waar ik tot nu toe mee gewerkt heb.\\

Omdat ik intensief gewerkt heb aan het ontwerp van de
UPPAAL-modellen heb ik ook op het gebied van specificeren veel
nieuwe dingen geleerd, ik was hiermee namelijk nog niet eerder in
aanraking gekomen.\\

Met de presentatie heb ik me niet beziggehouden, maar ik heb wel al
een presentatie gedaan voor modelleren, dat ik in blok D gevolgd heb
in het kader van de dubbele propedeuse die ik doe. Ik denk dan ook
dat dat niet echt een probleem zal zijn.\\

Wel heb ik het nodige geleerd van verslaglegging, ik heb namelijk
het verslag van het programmaontwerp grotendeels geschreven. De
verslaglegging daarvan is toch wel anders dan andere verslagen die
ik tot nu toe geschreven heb.\\

Deze OGO is in mijn ogen dus goed verlopen en ik heb er veel van
geleerd.//
