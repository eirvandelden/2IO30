\subsection{Persoonlijke evaluatie van Etienne van Delden}

    Dit was voor mij de derde ogo, met als grootste verschil dat we twee
    dubbele P-ers en twee minoren in het groepje hadden. Dit maal kwamen
    onze roosters dus niet overeen. Hierdoor moesten we af en toe door werken met minder mensen als normaal.
    Ook verschilde de ervaringen met (OGO) groepsprojecten, waardoor
    we in het begin een moeilijke start hadden. Sommige dingen die voor Coen en mij logisch waren,
    zoals verslagen op studyweb zetten, waren voor de anderen nog niet logisch.
    Gelukkig was alles goed gekomen, toen we duidelijker afspraken maakten, alles goed gingen
    doorspreken en ikzelf aangesteld werd als quality-manager. \\

    Ik heb mij vooral bezig gehouden met het hardware-model. Ik heb
    meegeholpen aan de bouw ervan en ik heb de scart-interface
    grotendeels aangesloten. Het aansluiten van de scart-interface om het
    processorbord en het model makkelijk aan te sluiten was mijn idee. Hoewel het
    zeker voordelen heeft, hebben we hierdoor wel veel tijd
    verloren. Het aansluiten van alle 24 kabels duurde langer als verwacht \'e er waren een aantal fouten gemaakt,
     Gelukkig is alles nog gelukt met de scart aansluiting. \\

    Ook was ik verantwoordelijk voor de presentatie. Hoewel ik op
     moment van schrijven nog niets kan zeggen over de presentatie
      op de laatste ogo dag, samen met de
    demonstratie, had ik de eindpresentatie wel al gebruikt bij de
    presentatie training. Coen van der Wel en ikzelf hebben daar de
    eindpresentatie gegeven, waarbij die als erg goed werd
    beoordeeld.`\\

    Hoewel ik vrij weinig code heb geschreven, heb ik wel veel
    geleerd over de problemen die onstaan bij het maken van een
    embedded systeem. Voor het eerst hebben we eerst een model in
    UPPAAL gemaakt, van zowel het hardware als het software model,
    en pas daarna gewerkt aan de software zelf. We hebben daarbij
    ondervonden dat dit enorm helpt, omdat je dan een richtlijn hebt
    voor hoe je code moet worden. \\

    Qua verantwoordelijkheid wou ik me meer op de achtergrond houden,
    uit vorige ogo's heb ik gemerkt dat ik soms me teveel
    met anderen bemoei. Dit was toch niet helemaal gelukt. Door de
    moeilijke start werd ik uiteindelijk aangesteld als
    quality-manager, waarbij ik de verantwoordelijkheid draag voor
    de ingeleverde documenten. Mijn taak resulteerde uiteindelijk
    tot externe agenda voor de anderen. Ik hoefde zelf niet
    opdrachten te geven, maar mij werd gevraagd wat er moest
    gebeuren. Dit maakte het voor mij prettiger, omdat ik niet opdrachten hoef uit te geven
    , maar ook voor de anderen die een vraagbaak haddden en iemand die wist wat er allemaal moest gebeuren. \\
