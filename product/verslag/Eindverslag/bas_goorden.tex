\subsection{Persoonlijke evaluatie van Bas Goorden}

Dit was ook voor mij de eerste OGO. De major die ik volg is
Technische Wiskunde en dit was \'{e}\'{e}n van de vakken die ik in
mijn minor gekozen heb. \\

Bij wiskunde hebben we intussen toch al een behoorlijk aantal
Modelleren-projecten gedaan. Hierdoor was ik bang dat ik aan dit
vak, op het gebied van samenwerken, weinig zou hebben. Het feit dat
de groepen bij dit vak uit zes in plaats van uit twee personen
bestaan had echter zijn gevolgen. Bij Modelleren weet je namelijk
altijd precies wat er gebeurt en als je zelf niks doet gebeurt er
ook niks. Het was daarom zaak om in de gaten te houden waar de rest
van de groep mee bezig was en om daar vragen over te stellen als zij
iets gedaan hadden wat ik nodig had. Daarnaast ben je, als je deel
uitmaakt van een groep van twee personen, zelf betrokken bij en
verantwoordelijk voor alle belangrijke beslissingen. Bij OGO was dit
niet zo, onze groep bestond namelijk voornamelijk uit mensen die
geen enkel probleem hebben met het nemen van belangrijke
beslissingen. Hier heb ik mij aan aangepast, door mezelf niet bij
alle belangrijke beslissingen te betrekken. Dit viel mij minder
zwaar dan ik verwacht had. Ik heb bij dit OGO-project dus eens een
andere manier van samenwerken meegemaakt en dat heeft zeker een
bijdrage geleverd aan mijn vaardigheden op dat gebied.\\

Wat natuurlijk ook belangrijk is, is dat ik veel geleerd heb wat de
informatica zelf betreft. Ik heb mij eigenlijk een beetje met alle
belangrijke onderdelen van dit project beziggehouden. Eerst met het
bouwen van het hardware-model, vervolgens met het maken van een
programmaontwerp in UPPAAL, daarna met het schrijven van een aantal
subroutines en tot slot met het verslag. Vooral van het maken van
het programmaontwerp en van het schrijven van code heb ik veel
geleerd. Met de presentatie heb ik mij helemaal niet beziggehouden,
maar bij Modelleren heb ik al een behoorlijk aantal presentaties
gehouden, dus ik denk niet dat dat erg is.\\

Al met al heb ik er dus veel van geleerd. Het is niet meer dan de
kers op de taart, maar toch het vermelden waard, dat ik het
onderwerp van het project interessant vond en dat de samenwerking
binnen de groep goed was. Al met al was dit dus een geslaagde OGO.\\
