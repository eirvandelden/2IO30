\subsection{Persoonlijke evaluatie van Coen van der Wel}

    Persoonlijk vond ik dit een geslaagde OGO. Vanwege een kleine
    miscommunicatie in het begin was er even een probleem met het
    samenstellen van de groepen, maar uiteindelijk werd de groep
    samengesteld uit 3 van mijn vrienden en 2 minor-studenten, waarmee
    ik inmiddels ook goed overweg kan. Ik denk dat dat zeker heeft
    bijgedragen aan het resultaat, want een goede relatie met collega's
    is ook echt belangrijk.\\
    \\
    Verder heb ik veel geleerd over pneumatiek en electrische
    interfaces, ik heb zeker mijn deel bijgedragen aan het uitwerken van
    deze twee dingen. Ook heb ik mijn LEGO \texttrademark \ ervaring uit kunnen
    breiden door het bouwen van het model met Fischertechnik \texttrademark \. De
    beperktheden van deze techniek leidden tot een meer gecompliceerd
    model, maar die uitdaging is juist leuk. Het programmeren van het
    model en het kennismaken met een embedded systeem is uiteraard ook
    een belangrijk aspect geweest en heb ik met veel plezier gedaan.\\
    \\
    De presentatie heeft Etienne grotendeels verzorgd, maar hierbij heb
    ik toch zeker geen onbelangrijke rol gespeeld. De
    presentatietraining was voor mij niet echt nuttig, ik heb er namelijk
    niet erg veel van geleerd. Het rapporteren en verslagleggen van
    (tussen)resultaten was zoals bij de vorige OGO's een beetje een blok
    aan het been, maar gaat naar mijn mening wel steeds beter. Ik ontken
    dan ook niet dat deze verslagen nuttig zijn, maar het spelen met het
    model is natuurlijk veel leuker ;-)\\
    \\
    Qua verantwoordelijkheden heb ik mezelf een beetje teruggetrokken. Bij
    vorige OGO's is mijn ervaring dat ik me met alles bemoei en dat uiteindelijk
    alles op mij neerkomt; bij deze OGO heb ik me meer afzijdig gehouden, met de
    keerzijde dat ik me niet erg nuttig heb gevoeld. Er moet vast een tussenweg
    zijn... dat als verbeterpunt voor mezelf voor een volgende OGO.\\
    \\
    Maar al met al vind ik het een geslaagde OGO!
