\documentclass[]{article}

% Use utf-8 encoding for foreign characters
\usepackage[utf8]{inputenc}

\usepackage[dutch]{babel}
\usepackage{graphicx}

% Setup for fullpage use
\usepackage{fullpage}

\title{Opbouw implementatie}
\author{Stef Sijben}

\begin{document}
\maketitle


\section{Algemeen}
In dit bestand wordt beschreven welke subroutines er nodig zijn in
het programma. Sommige van deze subroutines hebben parameters of
geven een waarde terug. In het geval van parameters wordt A of, in
het geval van meerdere parameters, R0..Rn gebruikt om deze door te
geven. Return waardes worden afhankelijk van hun type verschillend
teruggegeven: booleans komen in de carry bit, andere types in A.

\section{Low-level I/O}
De subroutines die hierbij horen worden doen heel simpele dingen,
als het in- of uitschakelen van een enkele motor, lamp of ventiel of
het uitlezen van een enkele sensor. Hier komt dus gewoon voor elk
hardwareapparaat een of meerdere subroutines.

\section{Belangrijke acties}

\subsection{Wafer verplaatsen}
Hierin komen een paar subroutines die zorgen dat de tweede lopende
band voor een bepaald aantal seconden of totdat een sensor iets
registreert in een bepaalde richting beweegt.

Daarnaast is er een subroutine die probeert een wafer van de eerste
band te halen.

\begin{itemize}
  \item MoveForwardT(s): laat de tweede lopende band voor s * factor
  seconden voorwaarts bewegen en stop dan.
  \item MoveBackwardT(s): idem, maar dan achteruit.
  \item MoveForwardS1, MoveForwardS2: Laat de tweede lopende band
  voorwaarts bewegen totdat lichtsensor 1 resp. 2 iets registreert
  en return true, of als de sensor na een vooraf vastgestelde tijd
  nog niets geregistreerd heeft, return false.
  \item MoveBackwardS1, MoveBackwardS2: idem, maar dan achteruit.
  \item LoadWafer: Probeer een wafer te laden van de eerste band.
  Als een wafer klaar ligt, return true.
  Als dit na een vastgestelde tijd nog niet gelukt is, return false.
\end{itemize}

\subsection{Luchtsluis}
Hierin komen subroutines voor het openen en sluiten van de deuren.

\begin{itemize}
  \item OpenDoor1, OpenDoor2: Controleer of de andere deur dicht is
  en zo ja, open de desbetreffende deur en controleer of hij open
  is. Return of dit gelukt is.
  \item CloseDoor1, CloseDoor2: Sluit een deur en controleer of ze
  echt dicht is. Return of dit gelukt is.
\end{itemize}

Doe een paar pogingen om een deur open of dicht te krijgen en return
pas false als het dan nog steeds niet gelukt is. In dat geval lijkt
de deur dus vast te zitten en moet het programma stoppen.

\subsection{UV lamp}
Hierin zit enkel de subroutine die het daadwerkelijk branden van de
wafer afhandelt.

\begin{itemize}
  \item Burn: Verander de Interrupt vector van de noodknop naar
  EMBurn, schakel de lamp in, wacht 2 seconden, schakel de lamp weer
  uit en zet de interrupt vector weer terug naar EMStandard.
\end{itemize}

\subsection{onderbrekingen}
Hierin komen interrupt handlers voor de noodstop en voor het
verliezen van wafers.

\begin{itemize}
  \item EMStandard: De normale handler voor het indrukken van de
  noodstop. Sla de toestand van alle uitvoerpoorten ergens op en
  schakel vervolgens alles uit. Wacht dan tot de noodstop opnieuw
  wordt ingedrukt en herstel dan alle uitvoerpoorten naar hun oude
  toestand.
  \item EMBurn: De handler voor het indrukken van de noodstop
  tijdens het branden van een wafer. In dat geval is de wafer
  mislukt en wordt de lamp uitgeschakeld. Bij het opnieuw indrukken
  van de knop wordt de wafer naar de afvalbak getransporteerd en
  vervolgens keren de deuren terug in hun uitgangspositie, waarna de
  volgende wafer geladen wordt.
  \item WaferLost: Registreer een verloren wafer, als er hierna nog
  steeds minder dan 5 verloren zijn, gaat het systeem naar de
  uitgangspositie en probeert de volgende wafer te laden. Als er wel
  5 verloren zijn crasht het systeem.
\end{itemize}


\end{document}
