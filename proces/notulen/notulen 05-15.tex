%% DOCUMENT DECLARATIE %%%%%%%%%%%%%%%%%%%%%%%%%%%
\documentclass[]{article}

%% USE PACKAGES %%%%%%%%%%%%%%%%%%%%%%%%%%%%%%%%%%
\usepackage[utf8]{inputenc}
\usepackage{fullpage}
\usepackage{amsmath}
\usepackage{amssymb}
\usepackage{boxedminipage}
\usepackage{listings}
\usepackage{minitoc}
\usepackage{ifpdf}
\ifpdf
    \usepackage[pdftex]{graphicx}
\else
    \usepackage{graphicx}
\fi

%% TITEL DECLARATIES %%%%%%%%%%%%%%%%%%%%%%%%%%%%%
\title{Notulen - OGO1.3, groep 7}
\author{OGO1.3, groep 7}
\date{27-03-07}

%% CRE�ER TITEL %%%%%%%%%%%%%%%%%%%%%%%%%%%%%%%%%%
\begin{document}
\ifpdf
    \DeclareGraphicsExtensions{.pdf, .jpg, .tif}
\else
    \DeclareGraphicsExtensions{.eps, .jpg}
\fi
\maketitle


%% START NOTULEN %%%%%%%%%%%%%%%%%%%%%%%%%%%%%%%%%
\section{Opening vergadering: 13:35 uur }
    \begin{itemize}
        \item Voorzitter: Coen van der Wel
        \item Notulist: Gijs Direks
        \item Afwezige leden: Etienne van Delden (afgemeld)
    \end{itemize}

    
\section{Mededelingen tutor}
    \begin{itemize}
        \item Vanaf deze week is HG 10.43 onze niewe ruimte.
        \item De presentatietraining is op donderdag 7 juni. Dit valt samen met de laatste vergadering. We bekijken later nog of we de vergadering verzetten of toch houden.
        \item De proefvoordrachten duren maar 10 minuten in plaats van 15.
    \end{itemize}

\section{Mededelingen leden}
    \begin{itemize}
        \item Sanne en Bas hebben dit blok op donderdags het vijfde en het zesde uur les. Hierdoor wordt de vergadering verzet naar ongeveer 15:15 uur.
    \end{itemize}

\section{Opdracht bespreking en verslaglegging}
    \begin{itemize}
        \item We gebruiken nu data in plaats van versienummers op onze verslagen. Dit is een beetje slordig - we zorgen ervoor dat versienummers nu standaard gebruikt worden.
        \item Assemblyprogramma: Als een loop voorkomt in het programma is het wenselijk dat de guards hiervan, alsmede de invarianten, ook in het UPPAAL programmaontwerp terugkomen.
        \item Ontwerpverslag: De ontwerpbeslissingen moeten beter over de hoofdstukken verdeeld worden, de laatste vier alinea's van blz. 22 kunnen terug naar hoofdstuk 2.
        \item Ontwerpverslag: De init-template is overbodig, we kunnen aannemen dat er een bepaalde begintoestand is (als we die vermelden). Eventueel kunnen we expliciet vermelden dat de init-template opgenomen is omdat het processorbord in een willekeurige toestand opstart.
        \item Ontwerpverslag: De labels van de componenten D1, D2, R1, R2 komen niet terug in de foto's of het UPPAAL-model.
        \item Ontwerpverslag: De start/noodknop hebben nu nog eenzelfde template. Licht toe waarom of verander dit.
        \item Ontwerpverslag: Het is onduidelijk of Button1 de noodknop of de readyknop is. Geef dit even aan. Daarnaast is bRelease2 afwezig - klopt dit?
        \item Ontwerpverslag: Het verschil tussen sensor en button: Sensor is 'stabiel' gemodelleerd, button is 'instabiel' gemodelleerd. Dit moet aangepast of toegelicht worden.
        \item Dipswitches: We moeten nog even nagaan of dhr. Moussavi het goed vindt als we een dipswitch als knop gebruiken, hij had dit namelijk expliciet verboden bij groep 8.
        \item Programmaontwerp: Routines die in het assemblyprogramma voorkomen worden als aparte templates verwerkt in UPPAAL, er is een soort maincontroller die specifieke kleine controllers aanroept.
    \end{itemize}

\section{Planning}
    \begin{itemize}
        \item Het programmaontwerp moet worden aangepast. Dit wordt vandaag (15 mei) nog ingeleverd.
        \item De systeemanalyse moet het liefst vandaag nog af. Als dit niet lukt wordt dit alsnog op maandag voor 12 uur ingeleverd.
        \item De implementatie wordt ingeleverd op vrijdag 25-05 voor 12 uur.
    \end{itemize}

\section{Sluiting}
    \begin{itemize}
        \item Sluiting vergadering om 14:15
    \end{itemize}

%% EIND DOCUMENT %%%%%%%%%%%%%%%%%%%%%%%%%%%%%%%%%
$\boxtimes$
\end{document}
