%% DOCUMENT DECLARATIE %%%%%%%%%%%%%%%%%%%%%%%%%%%
\documentclass[]{article}

%% USE PACKAGES %%%%%%%%%%%%%%%%%%%%%%%%%%%%%%%%%%
\usepackage[utf8]{inputenc}
\usepackage{fullpage}
\usepackage{amsmath}
\usepackage{amssymb}
\usepackage{boxedminipage}
\usepackage{listings}
\usepackage{minitoc}
\usepackage{ifpdf}
\ifpdf
    \usepackage[pdftex]{graphicx}
\else
    \usepackage{graphicx}
\fi

%% TITEL DECLARATIES %%%%%%%%%%%%%%%%%%%%%%%%%%%%%
\title{Notulen - OGO1.3, groep 7}
\author{OGO1.3, groep 7}
\date{03-05-07}

%% CRE�ER TITEL %%%%%%%%%%%%%%%%%%%%%%%%%%%%%%%%%%
\begin{document}
\ifpdf
    \DeclareGraphicsExtensions{.pdf, .jpg, .tif}
\else
    \DeclareGraphicsExtensions{.eps, .jpg}
\fi
\maketitle

%% START NOTULEN %%%%%%%%%%%%%%%%%%%%%%%%%%%%%%%%%
\section{Opening vergadering: 13:28 uur }
    \begin{itemize}
        \item Voorzitter: Coen van der Wel
        \item Notulist: Stef Sijben
        \item Afwezige leden: -
    \end{itemize}

\section{Bespreken van notulen vorige vergadering}
    \begin{itemize}
        \item Punt 6: Iedereen werkt iedere dinsdag aan OGO moet
        zijn: Iedereen is elke dinsdag aanwezig.
    \end{itemize}

\section{Mededelingen voorzitter}
    \begin{itemize}
        \item In subversion is revisie 100 bereikt.
        \item Groep 2 moet aangesproken worden op de rotzooi die ze
        achtergelaten hebben.
    \end{itemize}

\section{Mededelingen tutor}
    \begin{itemize}
        \item Eerstvolgende vergadering is 15 mei in de grote
        ruimte.
        \item Bas en Coen moeten hun logboek nog bijwerken.
        \item Het studentenoverleg is een week uitgesteld.
    \end{itemize}

\section{Mededelingen leden}
    \begin{itemize}
        \item -
    \end{itemize}

\section{Ontwerpverslag}
    \begin{itemize}
        \item Controller op abstract niveau wel opnemen in het
        ontwerpverslag i.v.m. de semantiek van het optreden van
        transities in netwerken van timed automata. In dit verslag
        is de matching tussen in- en uitgaande communicatie niet
        aanwezig.
        \item De noodknop moet ook gemodelleerd worden in het UPPAAL
        model. In de afhandeling hiervan moet het indrukken van de
        noodknop tijdens het belichten een uitzondering vormen op
        normaal.
        \item Het feit dat het programma crashed als er 5 wafers van
        de band zijn gevallen moet expliciet gemodelleerd worden,
        ook aan de kant van de ontvangst van het crash-signaal.
        \item In het ontwerpverslag mag expliciet de aanname worden
        gemaakt dat elk onderdeel van het ontwerp een nette
        begintoestand heeft.
        \item Er moet misschien onderscheid worden gemaakt tussen de
        ready button en de emergency button. Transities tussen de
        toestanden van de button kunnen beter met synchronisatie
        gemodelleerd worden.
        \item Er moet een inleiding toegevoegd worden.
        \item Er moet een opsomming toegevoegd worden van de namen van sensoren in
        UPPAAL met de functie die deze sensor in het hardwaremodel
        heeft.
        \item Het is misschien niet toegestaan om de DIP-switches op
        het processorbord te gebruiken als ready- en noodstopknop.
        \item De UPPAAL-templates horen niet in een appendix, de
        ontwerpbeslissingen in sectie 1.3 kunnen beter naar
        hoofdstuk 2 gaan.
        \item De eerste twee alinea's van sectie 1.4 kunnen beter
        naar hoofdstuk 2.
        \item Let op het gebruik van woorden als staten in een
        Nederlandse tekst.
    \end{itemize}

\section{Opdracht bespreking}
    \begin{itemize}
        \item In het programmaontwerp wordt de controller
        gemodelleerd als een netwerk van automaten, waarin elke
        subroutine als een template gemodelleerd wordt.
        \item In de systeemanalyse is het misschien mogelijk om
        toch voor alle paden te verifi\"eren dat een wafer binnen een
        bepaalde tijd uit het systeem is. Om dit te bereiken zou er
        in een variabele bijgehouden moeten worden of de noodstop
        tijdens het verwerken van die wafer is ingedrukt of niet. De
        verificatie zou dan gaan over alle paden waarin de noodstop
        niet is ingedrukt.
    \end{itemize}

\section{Planning}
    \begin{itemize}
        \item Het is belangrijk wat de feedback op het
        ontwerpverslag betekent voor het project. Moeten alleen
        verslagen aangepast worden of zijn er ook gevolgen voor
        andere delen van het project?
        \item Programmaontwerp snel inleveren
        \item Voor de systeemanalyse is uitstel mogelijk, deze is
        bijna af.
        \item Deze week zo veel mogelijk zaken nog inleveren.
        \item Coen, Stef en Bas gaan werken aan de systeemanalyse.
        \item Etienne en Gijs gaan werken aan de verbeterde versie
        van het ontwerpverslag.
        \item Sanne werkt het schema van voorzitters en notulisten
        bij.
    \end{itemize}

\section{Verslaglegging}
    \begin{itemize}
        \item Alles is duidelijk
    \end{itemize}

\section{W.V.T.T.K.}
    \begin{itemize}
        \item Door feedback op verslagen loopt de vergadering uit.
        Daarom zal de volgende keer niet alle commentaar toegelicht
        worden, alleen waar dat nodig is.
    \end{itemize}

\section{Sluiting}
    \begin{itemize}
        \item Sluiting vergadering om 14:08
    \end{itemize}

%% EIND DOCUMENT %%%%%%%%%%%%%%%%%%%%%%%%%%%%%%%%%
$\boxtimes$
\end{document}
