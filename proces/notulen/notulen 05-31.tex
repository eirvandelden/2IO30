%% DOCUMENT DECLARATIE %%%%%%%%%%%%%%%%%%%%%%%%%%%
\documentclass[]{article}

%% USE PACKAGES %%%%%%%%%%%%%%%%%%%%%%%%%%%%%%%%%%
\usepackage[utf8]{inputenc}
\usepackage{fullpage}
\usepackage{amsmath}
\usepackage{amssymb}
\usepackage{boxedminipage}
\usepackage{listings}
\usepackage{minitoc}
\usepackage{ifpdf}
\ifpdf
    \usepackage[pdftex]{graphicx}
\else
    \usepackage{graphicx}
\fi

%% TITEL DECLARATIES %%%%%%%%%%%%%%%%%%%%%%%%%%%%%
\title{Notulen - OGO1.3, groep 7}
\author{OGO1.3, groep 7}
\date{05-31-07}

%% CRE�ER TITEL %%%%%%%%%%%%%%%%%%%%%%%%%%%%%%%%%%
\begin{document}
\ifpdf
    \DeclareGraphicsExtensions{.pdf, .jpg, .tif}
\else
    \DeclareGraphicsExtensions{.eps, .jpg}
\fi
\maketitle

%% START NOTULEN %%%%%%%%%%%%%%%%%%%%%%%%%%%%%%%%%
\section{Opening vergadering: 15:31 uur }
    \begin{itemize}
        \item Voorzitter: Sanne Ernst
        \item Notulist: Etienne van Delden
        \item Afwezige leden: -
    \end{itemize}

\section{Bespreken van notulen vorige vergadering}
    \begin{itemize}
        \item -
    \end{itemize}

\section{Mededelingen voorzitter}
    \begin{itemize}
        \item -
    \end{itemize}

\section{Mededelingen tutor}
    \begin{itemize}
        \item Presentatie training moet voorbereid worden en moet ongeveer 10 minuten
    \end{itemize}

\section{Mededelingen leden}
    \begin{itemize}
        \item AP\footnote{AP: Actie Punt} Coen: Lokaal 10.46 lijkt niet gereserveerd te zijn, gelieve de heer Moussavi of Bloo te mailen of wij dit lokaal mogen reserveren op dinsdag- en donderdagmiddag. Het huidige lokaal heeft continu de zon erop, de blinds kunnen niet dicht en er zijn geen ramen die open kunnen.
    \end{itemize}

\section{Opdracht bespreking}
    \begin{itemize}
        \item Het hardware model is bijna af
        \item Er moet een begin gemaakt worden aan het eindverslag
        \item Er moet een begin gemaakt worden aan de presentatie
    \end{itemize} 


\section{Verslaglegging}
    \begin{itemize}
        \item Programmaontwerp moet worden opgenomen in het eindverslag
        \item Blz 4: templates maken gebruiken van templates uit het ontwerpverslag; let op de edits aan het ontwerpverslag in het eindverslag.
        \item Systeem Analyse: \\
        Blz 9: lost5?, crash! je kan niet zien waar het verstuurd respectievelijk ontvangen wordt, het staat niet in het ontwerp verslag.
        \item Blz11: wederom crash!
        \item Managecrash template kan niet in het ontwerpverslag worden teruggevonden
        \item In het main programma staat geen link naar de noodstop, w\'el in de figuren 2.1 tot en met 2.6
        \item Tip: tijdens het testen, pak een wafer weg om te testen of dit gedetecteerd wordt
    \end{itemize}

\section{Planning}
    \begin{itemize}
        \item AP Bas: Eindverslag skelet (liever met zoveel mogelijk tekst ) inleveren bij de heer Veltkamp op z'n laatst op maandag 04 juni, om 12:00 uur.
        \item Tip: Test resultaten van het model opnemen in het eindverslag
        \item AP Bas: Vorige verslagen aanpassen voor in het eindverslag
        \item AP Etienne: beginnen aan de presentatie
    \end{itemize}



\section{Sluiting}
    \begin{itemize}
        \item Sluiting vergadering om 16:07
    \end{itemize}

%% EIND DOCUMENT %%%%%%%%%%%%%%%%%%%%%%%%%%%%%%%%%
$\boxtimes$
\end{document}
