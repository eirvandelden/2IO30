%% DOCUMENT DECLARATIE %%%%%%%%%%%%%%%%%%%%%%%%%%%
\documentclass[]{article}

%% USE PACKAGES %%%%%%%%%%%%%%%%%%%%%%%%%%%%%%%%%%
\usepackage[utf8]{inputenc}
\usepackage{fullpage}
\usepackage{amsmath}
\usepackage{amssymb}
\usepackage{boxedminipage}
\usepackage{listings}
\usepackage{minitoc}
\usepackage{ifpdf}
\ifpdf
    \usepackage[pdftex]{graphicx}
\else
    \usepackage{graphicx}
\fi

%% TITEL DECLARATIES %%%%%%%%%%%%%%%%%%%%%%%%%%%%%
\title{Notulen - OGO1.3, groep 7}
\author{OGO1.3, groep 7}
\date{06-07-07}

%% CRE�ER TITEL %%%%%%%%%%%%%%%%%%%%%%%%%%%%%%%%%%
\begin{document}
\ifpdf
    \DeclareGraphicsExtensions{.pdf, .jpg, .tif}
\else
    \DeclareGraphicsExtensions{.eps, .jpg}
\fi \maketitle

%% START NOTULEN %%%%%%%%%%%%%%%%%%%%%%%%%%%%%%%%%
\section{Opening vergadering: 15:30 uur }
    \begin{itemize}
        \item Voorzitter: Sanne Ernst
        \item Notulist: Bas Goorden
        \item Afwezige leden: Etienne van Delden en Coen van der Wel
    \end{itemize}

\section{Bespreken van notulen vorige vergadering}
    \begin{itemize}
        \item Het programma ontwerp moet in het eindverslag.
    \end{itemize}

\section{Mededelingen tutor}
    \begin{itemize}
        \item Er staat een rapportage-sjabloon voor het eindverslag
        op studyweb.
    \end{itemize}


\section{Mededelingen leden}
    \begin{itemize}
        \item Bas en Sanne hebben tijdens de presentatie van de
        groepen 8 t/m 11 college, en gaan daarom aan dhr Mousavi
        vrijstelling hiervoor vragen.
    \end{itemize}


\section{Verslaglegging}
    De volgende opmerkingen betreffen het eindverslag.
    \begin{itemize}
        \item Hoofdstuk 2 moet nog worden ingevoegd.
        \item In hoofdstuk 3 moet het bedradingsschema en de
        beluchtingsschema komen te staan.
        \item In hoofdstuk 4 in paragraaf 4.2 in de tweede alinea
        moet in plaats van naar het ontwerpverslag naar hoofdstuk 2
        worden verwezen.
        \item Modellereen moet in modelleren veranderd worden.
        \item In hoofdstuk 4 moet meer naar hoofdstuk 2
        terugverwezen worden, onder andere om te laten zien waar het signaal lostfive vandaan komt.
        \item In hoofdstuk 4 in paragraaf 4.3 moet in plaats van
        naar het vorige hoofdstuk naar de vorige sectie, namelijk
        4.2, worden verwezen.
        \item In figuur 4.1 tot en met 4.6 staan emStart en
        emStop, maar deze ontbreken in figuur 4.7. Hierdoor lijkt het erop dat er in 4.7 geen noodstop kan plaatsvinden.
        \item Beschrijven wat er gebeurt als er tijdens de HandleFallof een wafer
        van de band valt.
        \item Het is goed als we in de bijlage van het verslag de
        assembly code zetten.
        \item Tussen hoofdstuk 5 en 6 moet een
        testbeschrijving/testresultaten komen. Dit hoeft dan dus
        niet meer in de evaluatie komen te staan.
        \item De logboeken mogen in de bijlage.
        \item De conclusie is nog wat mager. We kunnen de bereikte
        leerdoelen erin zetten, zoals op bladzijde 3 van de projectwijzer in paragraaf 1.1 staat.
    \end{itemize}

\section{Planning}
    \begin{itemize}
        \item De definitieve versie van het eindverslag moet
        uiterlijk donderdag 14 juni worden ingeleverd bij dhr Mousavi
        en bij onze tutor. Ook moet deze definitieve versie op
        studyweb komen te staan.
    \end{itemize}



\section{Sluiting}
    \begin{itemize}
        \item Sluiting vergadering om 15:55
    \end{itemize}

%% EIND DOCUMENT %%%%%%%%%%%%%%%%%%%%%%%%%%%%%%%%%
$\boxtimes$
\end{document}
